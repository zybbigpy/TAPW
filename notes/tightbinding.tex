% ****** Start of file apssamp.tex ******
%
%   This file is part of the APS files in the REVTeX 4.2 distribution.
%   Version 4.2a of REVTeX, December 2014
%
%   Copyright (c) 2014 The American Physical Society.
%
%   See the REVTeX 4 README file for restrictions and more information.
%
% TeX'ing this file requires that you have AMS-LaTeX 2.0 installed
% as well as the rest of the prerequisites for REVTeX 4.2
%
% See the REVTeX 4 README file
% It also requires running BibTeX. The commands are as follows:
%
%  1)  latex apssamp.tex
%  2)  bibtex apssamp
%  3)  latex apssamp.tex
%  4)  latex apssamp.tex
%
\documentclass[%
%reprint,
%superscriptaddress,
%groupedaddress,
%unsortedaddress,
%runinaddress,
%frontmatterverbose, 
%preprint,
%preprintnumbers,
%nofootinbib,
%nobibnotes,
%bibnotes,
 amsmath,amssymb,
% aps,
%pra,
%prb,
rmp,
%prstab,
%prstper,
%floatfix,
]{revtex4-1}

\usepackage{graphicx}% Include figure files
\usepackage{dcolumn}% Align table columns on decimal point
\usepackage{bm}% bold math
%\usepackage{hyperref}% add hypertext capabilities
%\usepackage[mathlines]{lineno}% Enable numbering of text and display math
%\linenumbers\relax % Commence numbering lines

%\usepackage[showframe,%Uncomment any one of the following lines to test 
%%scale=0.7, marginratio={1:1, 2:3}, ignoreall,% default settings
%%text={7in,10in},centering,
%%margin=1.5in,
%%total={6.5in,8.75in}, top=1.2in, left=0.9in, includefoot,
%%height=10in,a5paper,hmargin={3cm,0.8in},
%]{geometry}

\def \bkbar {\bar{\mathbf{k}}}
\def \bG    {\mathbf{G}}
\def \e     {\mathrm{e}}
\def \I     {\mathrm{I}}
\def \J     {\mathrm{J}}
\def \i     {\mathrm{i}}
\def \bR    {\mathbf{R}}
\def \bL    {\mathbf{L}}
\def \btau  {\bm{\tau}}
\def \bH    {\mathbf{H}}

\begin{document}

\preprint{APS/123-QED}

\title{Tight Binding Model for Twist Bilayer Graphene}% Force line breaks with \\
\thanks{A footnote to the article title}%

\author{Wangqian Miao}

\date{\today}% It is always \today, today,
             %  but any date may be explicitly specified



%\keywords{Suggested keywords}%Use showkeys class option if keyword
                              %display desired
\maketitle

%\tableofcontents



% The \nocite command causes all entries in a bibliography to be printed out
% whether or not they are actually referenced in the text. This is appropriate
% for the sample file to show the different styles of references, but authors
% most likely will not want to use it.
\nocite{*}



\bibliography{apssamp}% Produces the bibliography via BibTeX.

\section{Symbol Convention}

\section{The Tight Binding Hamiltonian for Twist Bilayer Graphene}

The tight binding Hamiltonian written in the second quantization language is

\begin{equation}
\begin{aligned}
\hat{H} = \sum_{\bkbar} \sum_{\alpha n, \beta m} H_{\alpha n, \beta m} C_{\alpha}^\dagger (\bkbar+ \bG_n)C_{\beta}(\bkbar+\bG_m).
\end{aligned}
\end{equation}

The basis wave function can be expressed as a summation of planewave:

\begin{equation}
\psi_{\alpha, n}(\bkbar) = \frac{1}{\sqrt{N_\mathrm{m} N_\mathrm{a}}}\sum_{\I, i} \e^{\i(\bkbar+\bG_n)\bR_{\I i \alpha}},
\end{equation}

And

\begin{equation}
\bR_{\I i \alpha} = \bL_{\I} + \btau_{i \alpha}.
\end{equation}

The matrix element of the Hamiltonian should be written as:

\begin{equation}
\begin{aligned}
H_{\alpha n, \beta m} &= \frac{1}{N_\mathrm{m} N_\mathrm{a}} \sum_{\I \J,ij} t(\bR_{\I i \alpha}-\bR_{\J j \beta})
\e^{-\i(\bkbar+\bG_n)\bR_{\I i \alpha}} \e^{\i(\bkbar+\bG_m)\bR_{\J j \beta}} \\
&= \frac{1}{N_\mathrm{m} N_\mathrm{a}} \sum_{\I \J,ij} t(\bR_{\I i \alpha}-\bR_{\J j \beta})
\e^{-\i(\bkbar+\bG_n)\cdot(\bL_{\I} +\btau_{i\alpha})} \e^{\i(\bkbar+\bG_m)\cdot(\bL_{\J} +\btau_{j\beta})} \\
&=\frac{1}{N_\mathrm{m} N_\mathrm{a}} \sum_{\I \J,ij} \e^{-\i \bG_n \btau_{i\alpha}} \e^{-\i \bkbar (\bL_\I -\bL_\J+\btau_{i\alpha}-\btau_{j\beta})} 
t(\bL_\I -\bL_\J+\btau_{i\alpha}-\btau_{j\beta}) \e^{\i \bG_m \btau_{j\beta}}\\
\text{cutoff: $\left<\I, \J\right>$,}\,\,&= \frac{1}{N_\mathrm{m} N_\mathrm{a}} \sum_{\I ,ij} \e^{-\i \bG_n \btau_{i\alpha}} \e^{-\i \bkbar (\bar{\btau}_{i\alpha, j\beta})} 
t(\bar{\btau}_{i\alpha, j\beta}) \e^{\i \bG_m \btau_{j\beta}} \\
\text{$\frac{1}{N_{\mathrm{m}}}\sum_{\I} =1$,}\,\,&=\frac{1}{N_\mathrm{a}} \sum_{ij} \e^{-\i \bG_n \btau_{i\alpha}} \e^{-\i \bkbar (\bar{\btau}_{i\alpha, j\beta})} 
t(\bar{\btau}_{i\alpha, j\beta}) \e^{\i \bG_m \btau_{j\beta}}. \\
\end{aligned}
\end{equation}

Tricky here, we write down the matrix form of the Hamiltonian,
\begin{equation}
\begin{aligned}
\bH_{\alpha\beta} &= \mathbf{X}^\dagger_{\alpha} \mathbf{T}_{\alpha\beta} \mathbf{X}_{\beta} \\
\bH &= \oplus_{\alpha \beta} \bH_{\alpha\beta}
\end{aligned}
\end{equation}
The above equation describes a block matrix multiplication. $\alpha,\beta$ runs in $[\mathrm{A}_1,\mathrm{B}_1, \mathrm{A}_2, \mathrm{B}_2]$

The matrix element:

\begin{equation}
\begin{aligned}
(\mathbf{X_\alpha}^\dagger)_{n,i}  &=\e^{\i\bG_n\btau_{i\alpha}},\\
(\mathbf{T_{\alpha \beta}})_{i, j} &= \e^{-\i \bkbar (\bar{\btau}_{i\alpha, j\beta})} 
t(\bar{\btau}_{i\alpha, j\beta}).
\end{aligned}
\end{equation}






\end{document}
%
% ****** End of file apssamp.tex ******